\documentclass[a4paper]{report}
\usepackage[utf8]{inputenc}
\usepackage[portuguese]{babel}
\usepackage{hyperref}
\usepackage{a4wide}
\hypersetup{pdftitle={UMCarroJá},
pdfauthor={José Ferreira, Joao Teixeira},
colorlinks=true,
urlcolor=blue,
linkcolor=black}
\usepackage{subcaption}
\usepackage[cache=false]{minted}
\usepackage{listings}
\usepackage{booktabs}
\usepackage{multirow}
\usepackage{appendix}
\usepackage{tikz}
\usepackage{authblk}
\usetikzlibrary{positioning,automata,decorations.markings}

\begin{document}

\title{UMCarroJá\\ 
\large Grupo Nº 48}
\author{José Ferreira (A83683) \and Joao Teixeira (A85504)}
\date{\today}

\begin{center}
    \begin{minipage}{0.75\linewidth}
        \centering
        \includegraphics[width=0.4\textwidth]{eng.jpeg}\par\vspace{1cm}
        \vspace{1.5cm}
        \href{https://www.uminho.pt/PT}
        {\color{black}{\scshape\LARGE Universidade do Minho}} \par
        \vspace{1cm}
        {\color{black}{\scshape\Large Programacao Orientada a Objetos}} \par
        \vspace{1.5cm}
        \maketitle
    \end{minipage}
\end{center}

\tableofcontents

\pagebreak

\chapter{Introdução}

O objetivo deste projeto é construir um sistema de aluger de carros,
inspirado no servico de aluguer de casas \textit{Airbnb}, onde um cliente
pode alugar um carro, onde ele mesmo e o condutor, para fazer a deslocacao
que pretende, ou disponibilizar as suas viaturas para alugar. Para a realizacao
deste projeto vamos aplicar conhecimentos adquiridos nas aulas da U.C. de 
Programacao Orientada a Objetos.
Ao longo deste relatório vamos descrever a nossa abordagem a este problema.

\chapter{Classes}\label{chap:api}

\section{Modelo}

\subsection{UMCarroJa}

Esta e a classe onde esta contida toda a informacao sobre utilizadores,
carros e alugueres. E tambem a grande ponte de comunicacao com o exterior
do modelo, permitindo assim que nao haja interacao direta do exterior com
as classes.

\subsection{User}

Esta e a classe com a informacao contida por qualquer user do sistema,
e metodos comuns tanto aos clientes como aos owners.

\subsection{Cliente}

Esta classe e referente ao cliente que pode criar alugueres, contendo
esta um Ponto, correspondete a posicao em que se encontra e alugueres
que ainda nao foram avaliados.

\subsection{Owner}

Esta classe e relativa ao utilizador que tem os seus carros para aluguer,
e este tem informacao sobre os carros que possui e tambem os alugueres
que ainda nao avaliou

\subsection{Car}

Esta classe representa uma viatura, onde tem todas as suas informacoes,
desde autonomia, quem e o seu proprietario, marca e matricula.

\chapter{Arquitetura e Solução do Projeto}

\section{Manutenção de Artigos}

A Manutenção de Artigos trata de manter artigos, respetivos preços
e nomes. Para manter essa informação usa dois ficheiros, \textit{strings}
e \textit{artigos}. O utilizador consegue adicionar e modificar artigos
através do stdin. Embora a manutenção de vendas seja independente dos 
restantes módulos, este comunica com o Servidor de Vendas, caso este esteja
a correr, de forma a manter a sincronização  da informação sobre o preço 
e numero de artigos existentes em ambos os lados. Para manter o tamanho
do ficheiro de \textit{strings} o mais pequeno possível, foi implementado
um compactador para esse ficheiro, garantindo assim que não existem grandes
volumes de dados desatualizados.\\
Aqui também é possível correr o agregador concorrente sobre o ficheiro 
de vendas.

\section{Servidor de Vendas}

O Servidor de Vendas trata dos pedidos de todos os clientes, e trata de 
manter informação sobre o stock de cada artigo. Para manter essa informação
utiliza o ficheiro \textit{stocks}. O Servidor corre em background, consistindo
em três processos distintos que tratam de tarefas diferentes:
\begin{itemize} 
    \item Sincronização com a Manutenção de Artigos
        (preços e criação de novos artigos), com a finalidade de manter,
        tanto o ficheiro de stocks, como a cache de preços sempre atualizada.
    \item Conter a cache de preços.
    \item Comunicação e tratamento dos pedidos dos clientes.
\end{itemize}
Toda a comunicação com o servidor é feita com recurso a \textit{Named pipes},
existindo um pipe para os clientes efetuarem pedidos e um para o Manutenção
de Artigos enviar as alterações que ocorram da parte dele.

\section{Cliente de Vendas}

O Cliente de Vendas interage com o utilizador a partir do \textit{stdin} e 
\textit{stdout}, e interage com o Servidor de Vendas enviando informação
para o pipe de entrada que o Servidor cria, e recebe a resposta deste
através de um pipe que ele cria, tendo como identificador o seu \textit{PID},
de forma a garantir que as informações não são lidas por outro cliente.
Para isto acontecer, para além do pedido, é escrito no pipe de leitura do
servidor o \textit{PID} do cliente.

\section{Agregador}

O agregador recebe input do \textit{stdin} linhas com o formato estipulado
para o ficheiro \textit{vendas}, calculando os totais de cada artigo,
devolvendo esses totais no mesmo formato pelo \textit{stdout}.

\chapter{Cache de Preços}

A cache de preços, parte integrante do Servidor de Vendas, existe para
evitar consultar o ficheiro de Artigos a cada pedido do cliente.
Para a manter sempre sincronizada com a informação que existe na parte da
Manutenção de artigos, optamos por implementar esta num processo distinto
do Servidor, afim de não causar atrasos nas respostas aos pedidos dos clientes.
Assim, quando o servidor necessita da informação sobre um preço de um artigo,
envia o pedido através do pipe criado para o efeito.

\chapter{Agregador Concorrente}

O Agregador Concorrente, implementado na Manutenção de Artigos, consiste
em correr o agregador sobre o ficheiro vendas, dividindo o em partes, sendo
capaz de agregar essas pequenas partes em paralelo e por fim agregar as 
partes mais pequenas todas, tornando assim a agregação deste ficheiro muito
mais rápida.

\chapter{Sincronização Artigos Stock}

Como a informação contida no ficheiro Stocks tem que estar sempre atualizada
com o ficheiro Artigos, ao ser criado o ficheiro Artigos, é guardada a 
informação de quando foi criado. Esta mesma marca temporária é transcrita para
o ficheiro Stocks na sua criação. Para assegurar que o ficheiro Stocks
tem a informação referente ao ficheiro Artigos mais atualizado, caso essa marca
difira entre ficheiros, o ficheiro Stocks é totalmente refeito. Também quando são
adicionados novos artigos, essas alterações são comunicadas ao Servidor, que trata
de actualizar o ficheiro. Quando o servidor não esta a correr, essa verificação
e feita ao arranque do mesmo, tendo assim a certeza que as referencias a artigos
do ficheiro Stocks são corretas e as mais recentes.

\chapter{Testes}

\section{Manutenção de Artigos}

Para testar a Manutenção de Artigos foram criados dois programas em Python.
O primeiro gera input em que as instruções dadas são válidas mas aleatórias.
O segundo gera um conjunto de instruções que é válido e é possível testar se estão
a ser bem interpretadas.
\begin{itemize} 
    \item Adiciona N produtos (passados como argumento), em que o nome é igual ao id do
        produto e o preço é igual a 0.
    \item Altera o nome de todos os produtos para 10 vezes o seu id.
    \item Altera o preço de todos os produtos para ser igual ao seu id.
\end{itemize}
Assim, para ver se o resultado é o esperado basta verificar se para um dado id
este tem preço igual a id e tem um nome igual a 10 vezes o id.

\section{Cliente de Vendas}

Para testar o Cliente de Vendas foram criados dois programas em Python que
seguem a mesma lógica dos criados para a Manutenção de Artigos.
O primeiro gera input em que as instruções dadas são válidas mas aleatórias.
O segundo gera um conjunto de instruções que é válido e é possível testar se estão
a ser bem interpretadas.
\begin{itemize} 
    \item Acede a todos os ids.
    \item Adiciona ao stock 100 artigos de cada artigo.
    \item Remove do stock 10 artigos de cada artigo.
\end{itemize}
Assim, para verificar se o resultado é o esperado podemos confirmar se para um dado
id o stock disponível é igual a 90.
Na mesma veia, se vários Clientes de Vendas correrem em concorrência, para validar
os resultados é possível verificar se a fórmula \textit{stock / 90 = N}, sendo o stock o stock
disponível para um dado artigo e N o número de clientes de vendas que foram corridos
em concorrência, é verdade.

\chapter{Conclusão}

Para concluir, conseguimos cumprir todos os requisitos propostos criando no processo um
sistema de gestão de stocks capaz de utilizar uma arquitetura \textit{Cliente, Servidor}
que suporta várias centenas de clientes.
Como trabalho futuro, gostaríamos de melhorar a relação entre o Manutenção de Artigos e
o Cliente de Vendas.

\end{document}
